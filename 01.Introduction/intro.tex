\clearpage
\pagenumbering{arabic}

\chapter{Introduction}

\section{Project Motivation}

Digital Electronics and circuit design are core fields in the study of Electronic Engineering, and are focused on the analysis and logical interpretation of digital signals, as well as the engineering of hardware that manipulates them in accordance with a desired logical function. A strong understanding of the fundamentals of digital electronics and circuit design serve as a foundation for multiple branches of study within Electronic Engineering. Therefore, it is vital that undergraduate students at Imperial College and other institutions have the best tools available to aid their study of these fundamental concepts.
One of the many challenges a first-year undergraduate student may face while learning Digital Electronic design is conceptualising relationships between inputs and outputs, and how these relationships relate to design specifications. At Imperial College London, EE students have the opportunity to gain a deeper insight into combinational logic through practical laboratory sessions, during which they create and simulate combinational circuits. The tools which they use must fulfil two criteria; firstly to provide an education-focused platform through which students can learn more about combinational logic and hardware design; secondly to be capable design tools in their own right which allow students to design and simulate complex logic.

Issie (Interactive Schematic Simulator with Integrated Editor) \cite{issie_repo}, an intuitive hardware design application, was developed at Imperial College London to address the lack of third-party programs that matched the above criteria. Issie is designed to be easy to use (requiring no user manual) and informative; visual cues and clear error messages guide students towards correct designs while individual step and waveform simulators allow students to vary inputs and see the effect on output values. This allows them to gain a better understanding of the hardware logic they have created. However, there is room for improvement.
In its current form, users of the applications can only view the outputs for one combination of inputs at a time. While this functionality is useful for exploring and verifying the behaviour of the circuit, it provides limited insight into the overall logic implemented by the circuit. Schematics are great at showing how the logic propagates through real hardware, but as the logic grows in complexity they can start to feel divorced from the original design specification. Therefore, an extension to Issie which provides an alternative approach to visualising combinational logic would likely improve students' learning experience. 

The goal of the curriculum is to first build up students' understanding of digital circuit design with schematics, and then eventually transition to the use of Hardware Description Languages (HDLs) in subsequent modules. Component-level schematics tend to focus on the propagation of digital signals through multiple components, while HDLs describe circuits at a behavioural level, focusing on the relationships between inputs and outputs. One could ask the following question: does Issie prepare students for this shift in how digital logic is perceived? The introduction of an alternative descriptive method of designing logic in Issie could act as a stepping stone between component-level and HDL-based digital logic design. Not only could this improve the learning experience, but it could also speed up design times as describing logic is often faster than designing it component-by-component.

One solution which may solve both aforementioned problems are truth tables. Truth Tables clearly show the relationship between all inputs and outputs, allowing users to infer the overall logic implemented by the hardware. Furthermore, by finding novel ways to display these truth tables and have users interact with them, their value addition to the learning and design experience can be boosted. Truth tables are also far better at describing logic compared to component-level schematics, and therefore may serve as a great intermediate step between schematics and complicated HDLs. Generating schematics from user-entered truth tables could also drastically reduce time spent designing hardware components which implement simple logic but require many gates and connections.

Thus, there is a strong case for finding alternative ways to visualise and input combinational logic which compliment Issie's existing schematic design framework, as such additions are likely to increase Issie's effectiveness as an education-focused hardware design platform. This will benefit students at Imperial College and other educational institutions.

\section{Project Definition}

The purpose of this project is to explore novel ways in which interactivity can be added to automatic schematic-derived truth tables, and how interactively generated truth tables can be used as a fast aid to design combinational logic. This is to achieve the overall aim of this project - to improve Issie in such a way that it is easier for students to understand the use of logic in digital design.
The deliverable will be integrated as an extension to the Issie application, with users being able to generate truth tables from the schematic and interact with them in ways that will augment their understanding of the logic they are designing and of Digital Electronics concepts in general. 
This project will conduct a short evaluation of Issie, highlighting the areas where it can be improved. While the primary focus of the project is on visualising combinational logic with interactive truth tables, the project will also seek to improve the overall user experience of Issie in other ways such as tweaking/redesigning elements of the UI or changing how information is communicated to users such that it is consistent and clear. 
In addition to improving the user experience for Issie, this project also aims to improve the developer experience wherever possible. Since its inception, maintainability and extensibility have been key to Issie \cite{marco_diss}; therefore the code contributed to Issie repository should be well-documented, readable, and interface well with existing code so that it is easy for future Issie developers to maintain and extend it. Further to this, if an appropriate opportunity arises then the project should also aim to reduce technical debt within the existing codebase.

\subsection{Core Principles of Issie} \label{subsec:principles}
As this project aims to improve Issie, any work done on this project should align with Issie's core principles. All features implemented in Issie must be:
\begin{enumerate}
    \item \textbf{Robust:} Software is robust when it is able to handle errors and behave correctly under exceptional circumstances, such as when supplied with erroneous inputs \cite{robust}. Issie in-fact goes a step further and notifies the user of the nature of the error.
    \item \textbf{Obvious:} The visual output given to the user should make it obvious what is happening with the schematic without the need for unnecessary explanation. Issie prefers to \textit{show not tell} in order to remain beginner-friendly.
    \item \textbf{Intuitive:} All functionality must be easy to expose to the user - there should be no need for detailed user guides as the UI for all functionality must be designed in a way such that users can intuitively learn how to use all of the application's features.
\end{enumerate}

In addition to these core principles, any extensions this project makes to Issie must also take into account the targeted users and the intended primary use case - teaching undergraduate students in a university laboratory while also enabling students to carry on where they left off at home. Thus, all new features must be cross-platform compatible and be suitable for students working in a laboratory and working alone at home. In conclusion, this project has two final deliverables. The first is an improved version of Issie, while the second deliverable consists of appropriate documentation of added features, and improvements to the documentation of existing features.



