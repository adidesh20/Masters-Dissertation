\clearpage
\pagenumbering{arabic}

\chapter{Introduction} \label{chap:intro}

\section{Project Motivation}

Digital Electronics and circuit design are core fields in the study of Electronic Engineering and are focused on the analysis and logical interpretation of digital signals, as well as the engineering of hardware that manipulates them in accordance with a desired logical function. A strong understanding of the fundamentals of digital electronics and circuit design serve as a foundation for multiple branches of study within Electronic Engineering. Therefore, it is vital that undergraduate students at Imperial College and other institutions have the best tools available to aid their study of these fundamental concepts.
One of the many challenges a first-year undergraduate student may face while learning Digital Electronic design is conceptualising relationships between inputs and outputs, and how these relationships relate to design specifications. At Imperial College London, EE students have the opportunity to gain a deeper insight into combinational logic through practical laboratory sessions, during which they create and simulate combinational circuits. The tools which they use must fulfil two criteria; firstly they must provide an education-focused platform through which students can learn more about combinational logic and hardware design; secondly they must be capable design tools in their own right which allow students to design and simulate complex logic.

Issie (Interactive Schematic Simulator with Integrated Editor) \cite{issie_repo}, an intuitive hardware design application, was developed at Imperial College London to address the lack of third-party programs that matched the above criteria. Issie is designed to be easy to use (requiring no user manual) and informative; visual cues and clear error messages guide students towards correct designs while individual step and waveform simulators allow students to vary inputs and see the effect on output values. This allows them to gain a better understanding of the hardware logic they have created. However, there is room for improvement.
In its current form, users of the application implement digital logic by building it component-by-component on a schematic diagram. Any syntactically correct digital circuit can be simulated using the \textit{Step Simulator}. In the Step Simulator, users specify values for each input to the digital logic, and can read the corresponding output values. Intermediate values can also be observed using \textit{Viewer} components. This functionality enables the user to easily verify their schematic with specific test cases, but lacks the ability to clearly summarise and verify the overall relationship between the inputs and outputs of the logic circuit. Users must therefore gain an overall understanding of the circuit through a combination of:
\begin{enumerate}
    \item Visually analysing the schematic to understand its logical function.
    \item Entering different input combinations into the Step Simulator and analysing the effect each change has on the outputs.
\end{enumerate}
As the implemented digital logic grows in complexity, the relationship between the inputs and outputs often becomes more obscure, and the schematic itself grows in size and can start to feel divorced from the specification. In such situations, the aforementioned method for understanding the logic circuit becomes less effective. To stop the schematic from getting too large and crowded Issie lets users define hierarchical \textit{Custom Components} which modularise the schematic and cut down on logic duplication. For example, an ALU may be implemented as a Custom Component within a CPU design schematic. This feature however, does not fully combat the issue of obscure relationships between inputs and outputs for complex circuits. Firstly, custom components that are not named clearly further obscure the logic function of the circuit. Secondly, due to their hierarchical nature, custom components can be nested inside other custom components, meaning that the user may have explore multiple layers of nested components before they can analyse the top-level schematic. This is a time-consuming exercise, requiring significant effort by the user. Therefore, there is significant value to adding functionality to Issie which allows users to better understand the relationships between inputs and outputs in digital logic circuits in a shorter amount of time.

One possible solution to this problem is automatically generating truth tables from the schematic. Truth tables exhaustively show the relationship between all inputs and outputs in an organised, persistent format. Inspecting cases in a truth table is far quicker than repeatedly changing values in the Step Simulator.
Furthermore, by investigating novel ways of presenting and interacting with these truth tables their value addition to the learning and circuit design experience in Issie can be boosted. For example, the ability to  present relationships inferred from the schematic in the truth table, or reduce an existing truth table with user-defined constraints, would provide the user with far more information than a simple simulation.

There is also merit in investigating the reverse; generating schematics from user-entered truth tables. This could reduce time spent designing hardware components which implement simple logic but require many gates and connections, as well as serve as a stepping stone between schematic design and HDL-based design.

Thus, there is a strong case for finding and implementing alternative ways to visualise (and possibly input) combinational logic in Issie to enable users to gain a better understanding of relationships in the logic, as well as the specification of the top-level design. If added in a way which compliments Issie's existing features, such enhancements are likely to increase Issie's effectiveness as an educational platform in addition to its capability as a digital logic design tool.
This will benefit students at Imperial College and other educational institutions.

\section{Project Definition}

The purpose of this project is to explore novel ways in which interactivity can be added to automatic schematic-derived truth tables, and how interactively generated truth tables can be used as a fast aid to design combinational logic. This is to achieve the overall aim of this project - to improve Issie in such a way that it is easier for students to understand the use of combinational logic in digital design.
The deliverable will be integrated as an extension to the Issie application, with users being able to generate truth tables from the schematic and interact with them in ways that will augment their understanding of the logic they are designing and of Digital Electronics concepts in general. 
This project will conduct a short evaluation of Issie, highlighting the areas where it can be improved. While the primary focus of the project is on visualising combinational logic with interactive truth tables, the project will also seek to improve the overall user experience of Issie in other ways such as tweaking/redesigning elements of the UI or changing how information is communicated to users such that it is consistent and clear. 
In addition to improving the user experience for Issie, this project also aims to improve the developer experience wherever possible. Since its inception, maintainability and extensibility have been key to Issie \cite{marco_diss}; therefore the code contributed to the Issie repository should be well-documented, readable, and interface well with existing code so that it is easy for future Issie developers to maintain and extend it. Further to this, if an appropriate opportunity arises, the project should also aim to reduce technical debt within the existing codebase.

\subsection{Core Principles of Issie} \label{subsec:principles}
As this project aims to improve Issie, any work done on this project should align with Issie's core principles. All features implemented in Issie must be:
\begin{enumerate}
    \item \textbf{Robust:} Software is robust when it is able to handle errors and behave correctly under exceptional circumstances, such as when supplied with erroneous inputs \cite{robust}. For simulations and text field inputs, Issie notifies the user of the nature of the error. User input, no matter how malformed, must never crash the application or lead to undefined behaviour.
    \item \textbf{Obvious:} The visual output given to the user should make it obvious what is happening without the need for unnecessary explanation. Issie prefers to \textit{show not tell} in order to remain beginner-friendly. For example, Issie uses colour-coded popups and highlights to draw user attention where it is needed and communicate events clearly.
    \item \textbf{Intuitive:} All functionality must be easy to expose to the user - there should be no need for detailed user guides as the UI for all functionality must be designed in a way such that users can intuitively learn how to use all of the application's features.
\end{enumerate}

In addition to these core principles, any extensions this project makes to Issie must also take into account the targeted users and the intended primary use case - teaching undergraduate students in a university laboratory while also enabling students to carry on where they left off at home. Thus, all new features must be cross-platform compatible and be suitable for students working in a laboratory and working alone at home.

In conclusion, this project has two final deliverables. The first is an improved version of Issie, while the second deliverable consists of appropriate documentation of added features, and improvements to the documentation of existing features.



