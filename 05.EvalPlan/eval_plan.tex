\chapter{Evaluation Plan}
This project has two main deliverables; an extended and improved version of Issie, and any appropriate documentation for the application. These deliverables will be evaluated against a series of metrics, both qualitative and quantitative.
\section{Evaluation against Requirements}
The bare minimum expectation of the project is that it \textbf{must} fulfil all Essential Requirements outlined in the Requirements Capture (Chapter \ref{chap:requirements}). Anything less than this would suggest that the project was not successful in achieving its goals. The only exception to this would be that if, during the development process, a feature outlined by a requirement was deemed to be unnecessary and therefore changed. The success of the project will also be measured on how many of the Desired Features are fulfilled, with the ideal scenario being that all requirements are satisfied. 

Most of the outlined requirements are features that the final deliverable should incorporate. Therefore, these requirements can simply be evaluated by opening the application and testing that they work as intended. The correctness of such features can be tested by opening existing Issie schematics featuring various circuits built by users, and checking if the added features perform as expected when used on those sheets. Additionally, testing will occur throughout the development process - each feature will be unit tested prior to its integration to Issie. 
In contrast to the type mentioned above, certain requirements cannot be objectively judged as complete, such as those related to intuitiveness of the UI. The subjective nature of such requirements mandates that their satisfaction be evaluated by multiple end-users. Therefore, the chosen evaluation methods for Requirements \textbf{E1.6}, \textbf{E1.7}, and \textbf{D3.1} will be a survey answered by volunteers who have spent time using the delivered application to perform a series of basic tasks.

\section{Overall Evaluation}
The overall aim of the project is to improve Issie as a digital electronics education tool and as a logic design application. The effectiveness of the features this project will add must be judged by those who stand to benefit from them - the user base. Therefore, the survey mentioned in the previous section will also include questions on if the newly added features helped them in designing the logic specified in the tasks. For one task in the survey the group will be split into two teams, with one using the original version of Issie and the other the one delivered by this project. The students will be given a series of custom components which implement increasingly complex, but still identifiable logic. The students will be asked to use Issie to identify which actual component the custom component implements using the tools their version of Issie provides them. The amount of time it takes them to reach an answer, and the correctness of that answer, will contribute to their score. The correlation between scores and Issie version will be observed and used to evaluate the version of Issie delivered by this project.