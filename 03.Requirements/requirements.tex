\chapter{Project Requirements} \label{chap:requirements}
This chapter specifies the comprehensive set of requirements for the updated version of Issie delivered by this project, as well as any accompanying documentation. These requirements build up to a more formal specification that is representative of the project aims, and takes into account the prior evaluation of the current version of Issie. Upon completion of the project, the deliverables will be evaluated against these requirements to ascertain the extent to which the project was successful.

\section{Requirements for Combinational Logic Visualisation}
The primary purpose of this project is to develop new ways for visualising combinational logic in Issie. These requirements specify the features that these novel visualisation methods should have, as well as constraints on their operation (e.g. required performance). Requirements can either be essential (\textbf{Ex}) or desirable (\textbf{Dx}). Essential requirements must be fulfilled for the project to be considered successful.
\subsection*{Essential Requirements}
\begin{itemize}
    \item[\textbf{E1.1}] Analyse a schematic containing \textbf{only combinational logic} and display a standard numeric truth table for that sheet.
    \medskip
    \item[\textbf{E1.2}] Analyse part of a schematic selected by the user containing \textbf{only combinational logic} and display a standard numeric truth table for that selection.
    \begin{itemize}
        \item[\textbf{E1.2.1}] New inputs and/or outputs should be created temporarily (if necessary) to feed inputs and/or read outputs from the selected logic.
        \item[\textbf{E1.2.2}] It must be clear which newly generated inputs/outputs shown in the truth table correspond to inputs/outputs into the selected logic.
    \end{itemize}
    \medskip
    \item[\textbf{E1.3}] Have a truth table generating algorithm which can handle:
    \begin{itemize}
        \item[\textbf{E1.3.1}] Multi-bit inputs and outputs. Any temporary inputs/outputs created while generating a truth table for a selected logic block must have correct widths.
        \item[\textbf{E1.3.2}] Custom Components (sub-sheets), including when they are part of selections.
        \item[\textbf{E1.3.3}] Displaying inputs, outputs, and viewers.
    \end{itemize}
    \medskip
    \item[\textbf{E1.4}] Give users an option, when possible, to reduce the truth table based on patterns in the logic (e.g. Don't Cares).
    \medskip
    \item[\textbf{E1.5}] Give users the option to filter the truth table by fixing input or output values.
    \medskip
    \item[\textbf{\textbf{E1.6}}] Truth Tables must be displayed in a clear and easy to understand format, and features involving truth tables (e.g. filtering, reducing etc.) must be presented in an intuitive way.
    \item[\textbf{\textbf{E1.7}}] Truth Table generation and reduction must take no longer than 4 seconds, with an ideal target of under 2 seconds.
    \item[\textbf{\textbf{E1.8}}] Graphical manipulation operations on the Truth Table, such as re-ordering rows, sorting etc. should appear instantaneous (i.e. take less than 100ms).
    \item[\textbf{E1.9}] Use the above features to give users a clearer insight into the digital logic on the schematic, or to further reinforce their existing understanding.
\end{itemize}

\subsection*{Desirable Features}
\begin{itemize}
    \item[\textbf{D1.1}] Generate and display algebraic truth tables.
    \begin{itemize}
        \item[\textbf{D1.1.1}] At minimum should at least support multiplexer and adder circuits.
        \item[\textbf{D1.1.2}] Preferably should have a rich set of algebraic operators with support for most circuits.
    \end{itemize}
    \medskip
    \item[\textbf{D1.2}] Provide an interactive truth table interface.
    \begin{itemize}
        \item[\textbf{D1.2.1}] Mousing over parts of the truth table could have effects on the schematic (e.g. annotations or highlighting).
        \item[\textbf{D1.2.2}] Users can rearrange order of columns/rows in the truth table.
        \item[\textbf{D1.2.3}] Users can sort the truth table in ascending and descending order.
    \end{itemize}
    \medskip
    \item[\textbf{D1.3}] Let the user access truth table related functionality without going through numerous steps \textbf{while} also keeping the number of buttons on the screen to a minimum to avoid cluttering the interface. 
    \medskip
\end{itemize}

% \section{Requirements for Logic Input}
% \subsection*{Essential Requirements}
% \begin{itemize}
%     \item[\textbf{E2.1}] Generate a correct Issie Custom Component using a user-supplied truth table.
%     \medskip
%     \item[\textbf{E2.2}] Users should be able to supply a truth table through Issie's GUI, or through a specified file.
% \end{itemize}

% \subsection*{Desirable Features}
% \begin{itemize}
%     \item[\textbf{D2.1}] Generate a correct Issie schematic using a user-supplied truth table, with the option for SOP and POS interpretations.
%     \begin{itemize}
%         \item[\textbf{D2.1.1}] Components in the generated schematic must be clearly spaced and laid out in a reasonable form.
%         \item[\textbf{D2.1.2}] Use intelligent analysis or Boolean Algebra to generate a schematic using as many components in Issie's catalogue as possible.
%     \end{itemize}
%     \medskip
%     \item[\textbf{D2.2}] If intelligent analysis is implemented, then use the same mechanism to simplify existing schematics.
% \end{itemize}

\section{Software/Documentation Quality Requirements}
In the process of adding new combinational logic visualisation methods to Issie, this project will also make significant additions/changes to the Issie codebase. This section highlights the requirements for the quality of the software contributed, along with the documentation for said software.
\subsection*{Essential requirements}
\begin{itemize}
    \item[\textbf{E2.1}] Deliver performant, working, bug-free code which adheres to Issie's code guidelines and other principles such as "MVU-ness".
    \medskip
    \item[\textbf{E2.2}] Write comments in the delivered code which adequately explain it such that it may be worked on in the future by other developers. 
    \medskip
    \item[\textbf{E2.3}] Deliver code that is easy to maintain for future developers.
    \medskip
    \item[\textbf{E2.4}] Provide any other necessary documentation.
\end{itemize}

\subsection*{Desired features}
\begin{itemize}
    \item[\textbf{D2.1}] Deliver a tweaked, or possibly partially redesigned UI which exposes all Issie features in a straightforward and intuitive way to the user, with a focus on extensibility (i.e. can the UI accommodate for future extensions to Issie?)
    \medskip
    \item[\textbf{D2.2}] Update the Issie website with information about any newly added features.
\end{itemize}

