\clearpage
\pagenumbering{roman}

\section*{\centering Final Report Plagiarism Statement}

I affirm that I have submitted, or will submit, an electronic copy of my final year project report to the provided EEE link.

I affirm that I have submitted, or will submit, an identical electronic copy of my final year project to the provided Blackboard module for Plagiarism checking.

I affirm that I have provided explicit references for all the material in my Final Report that is not authored by me, but is represented as my own work.

\newpage

\section*{\centering Acknowledgments}
First and foremost, I would like to thank my project supervisor, Dr Thomas Clarke, for his invaluable guidance and support throughout the duration of the Final Year Project, as well as introducing me to the F\fsharp programming language in my third year at Imperial College London -- something which led me to choose this project.

I am also deeply grateful to my parents. They have always inspired me to reach for new heights, while always supporting me so that I can do so without fear of falling.

Finally, I would also like extend my thanks to all previous Issie developers for contributing to this wonderful application and making it a solid platform for me to build on top of.

\newpage

\section*{\centering Abstract}
Issie (Interactive Schematic Simulator with Integrated Editor) is an education-focused digital electronics design platform used by students at Imperial College London, featuring an intuitive user-friendly UI and capable logic simulator. One of the many challenges a student may face when learning to design digital logic is conceptualising relationships between inputs and outputs in combinational logic, and how they relate to design specifications. This project extends the existing Issie application, exploring novel ways to communicate these logic relationships to the user. This is primarily achieved through the implementation of interactive automatic schematic-derived truth tables, which can be manipulated, reduced, and filtered to describe combinational logic relationships in numeric and algebraic forms. This involved the definition of an alternative formal language to Boolean algebra for representing combinational logic. 

Additionally, this project has also made changes to the top-level UI of the application to make the user experience more consistent and conducive to learning.
The extended functionality delivered by the project is effective and performant; this has been confirmed through a user experience survey. The novel visualisation methods have been tested on circuits designed by EE students at Imperial College London. Large digital logic circuits, such as the Arithmetic Logic Unit of an 8-bit ARM CPU, can be condensed from a schematic containing billions of possible input combinations to a short algebraic truth table describing a few dozen cases. 

\newpage

\tableofcontents

\listoftables

\listoffigures

\newpage
\chapter*{Table of Acronyms}

\begin{table}[!ht]
    \centering
    \begin{tabular}{|l|l|}
    \hline
        \textbf{Acronym} & \textbf{Definition} \\ \hline
        ALU & Arithmetic Logic Unit \\ \hline
        CLR & Common Language Runtime \\ \hline
        CSS & Cascading Style Sheets \\ \hline
        DC & Don't Care \\ \hline
        DOM & Document Object Model \\ \hline
        GUI & Graphical User Interface \\ \hline
        HTML & Hypertext Markup Language \\ \hline
        ISSIE & Interactive Schematic Simulator with Integrated Editor \\ \hline
        MVU & Model View Update \\ \hline
        UI & User Interface \\ \hline
        UWP  & Universal Windows Platform \\ \hline
        WBS & Work Breakdown Structure \\ \hline
        WPF & Windows Presentation Foundation \\ \hline
    \end{tabular}
\end{table}