\chapter{Conclusion and Further Work}
In conclusion, this project delivered an updated version of Issie, featuring interactive automatic schematic-derived truth tables which can be filtered, manipulated, and reduced using either Don't Care terms or Algebra. These truth tables offer users of Issie a novel way to visualise combinational logic relationships in their designs, aiding the design process and improving users' understanding of Digital Electronic design. With the majority of survey participants either agreeing or strongly agreeing that the added truth tables and algebraic expressions made it easier to understand the relationships between combinational inputs and outputs, it is evident that this novel visualisation technique is a valuable tool in Issie's arsenal for improving the learning experience for those learning digital logic design. Algebraic truth tables in particular have been very effective, condensing the millions of possible combinations in a complex ALU circuit into only 64 rows, which could then easily be further reduced by hand to specification comprising only eight lines. Through the collection of user feedback, it has been established that the novel logic visualisation methods added to Issie make it easier for users to better understand the relationships between inputs and outputs in combinational logic. All of the project aims and requirements have been met, meaning that this project can be considered successful.

This project has presented many unique and interesting challenges, and the process of overcoming them has been immensely rewarding. One of the initial challenges of the project was simply getting started with the implementation, as it involved working on an existing codebase which exceeded 20,000 lines of F\fsharp code. Therefore, prior to the implementation phase, a thorough analysis of the codebase was undertaken to create a solid platform of understanding upon which the new features could be built.

Writing the canvas correction algorithm, which enables users to generate truth tables for partial selections of a sheet, was also particularly challenging. From the outset, Issie has been effective at clearly describing errors to users, but has never taken steps to automatically correct these errors itself. As a result, all pre-existing simulation code is expected to take only syntactically correct canvas states. A major challenge with correcting a partial canvas state is that the user is free to make any selection possible. Therefore there is no guarantee of the form of the canvas state the algorithm may receive. Inferring the user's intentions from this unpredictable canvas state and generating a useful truth table with correct widths and informative IO labels was a complex process which posed both conceptual and implementation challenges.

Another task which was conceptually challenging was the design of the new formal language for algebra, in particular the definition of reduction rules. Some reduction rules, such as those for arithmetic simplification, appear simple on the surface but are far more difficult to generalise within a given system. Mixing Boolean algebra, fixed-width arithmetic, and other bus operations to create a novel algebraic system was in itself a stimulating task, and implementing methods to reduce these expressions was even tougher. However, the hard work undertaken to overcome these challenges has paid off; the delivered system is capable of simplifying complex circuits into succinct and informative algebraic truth tables.

\section{Possible Further Work}
While this project has undoubtedly been a successful endeavour, there are a few areas where its work could be extended to achieve an even better outcome. This section highlights some of these possible extensions.
\subsection{Algebraic Output Constraints} \label{subsec:algoutputcons}
As mentioned during the evaluation, output constraints can currently only be applied to numeric truth tables. One possible improvement to be implemented in the future could be the introduction of output constraints whose right-hand side is an algebraic expression. This would have two advantages. The first is that the outputs of the algebraic truth table could be successfully constrained -- only algebraic expressions which matched the output constraints would remain in the displayed table. The second is that algebraic constraints would also allow for a more intelligent filtering of numeric truth tables. For example, the output constraint $OUT = A + B$ would filter the table so that only rows in which the output was the sum of inputs $A$ and $B$ would be permitted. Such a feature would likely help users better analyse and understand the logic being designed.
\subsection{More Algebraic Reduction Rules}
Expanding the set of formal algebraic reduction rules would allow for the recognition of more complex constructs. One such example would be the recognition of ripple-carry adders. Currently, the algebraic truth table does recognise ripple-carry adders to some extent. For example, the output of a two-bit adder is:
\begin{align*}
    (A[1] + B[1] + carry(A[0] + B[0]))::(carry(A[0] + B[0]))
\end{align*}
However, ideally this would be simplified to $A + B$. The pattern to be checked for is recursive in nature, and would require a more formal definition prior to implementation in the future.
\subsection{Adding Algebra to the Step Simulator}
During the project, the Fast Simulation code was extended to support algebraic expressions. This was done so that algebraic truth tables could be implemented. The Step Simulator also uses the Fast Simulator to run simulations -- therefore adding support for algebraic inputs and outputs to the step simulator would only require a few changes, with the bulk of those being changes to the UI to allow an input to be toggled between algebra and numeric values. For the purposes of consistency, UI elements like the algebra selector popup could be re-used from the existing truth table codebase.

\section{User Guide}
Instructions on how to install Issie can be found on the Github page: \url{https://github.com/tomcl/issie}. This page also contains a link to the Issie website, which contains a guide on how to use the application.